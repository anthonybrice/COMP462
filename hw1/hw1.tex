\documentclass{abrice}

\title{Comp 462: Homework 1}
\author{Anthony Brice}

\begin{document}
\maketitle

\section{Exercise 2.1}

\begin{enumerate}[label=\alph*)]
\item Embedded -- The quality of being hidden.
\item Device driver -- A set of software functions that facilitate the use of
  an I/O port.
\item Top down -- A methodology for describing design as a cyclic process,
  beginning with a problem statement and ending up with a solution.
\item Call graph -- A graphical way to define how the software/hardware modules
  interact.
\item Data flow graph -- A block diagram of a system, showing the flow of
  information.
\item Successive refinement -- A technique to convert a problem statement into a
  software algorithm consisting of iteratively decomposing the task into a set
  of simpler subtasks.
\item Dynamic efficiency -- A measure of how fast a program executes via seconds
  or processor bus cycles.
\item Static efficiency -- The number of memory bytes a program requires in
  execution.
\item Real time -- A property of certain computer systems in which an upper
  bound is placed on the time required to perform the input-calculation-output
  sequence.
\end{enumerate}

\section{Exercise 2.3}

A requirement describes a specific parameter that a proposed system must
satisfy, like the ability to mount on a car dash. A specification is a detailed
parameter describing how a system should work, such as giving the exact size and
weight of a device. A constraint is a limitation within which a system must
operate, such as a restriction on available materials.

\section{Exercise 2.5}

Three operations software on a microcomputer embedded in a vending machine must
perform include: summing input currency, calculating and dispensing change, and
dispensing a user's specific selection among a variety.

\section{Exercise 7.1}

\end{document}

%%% Local Variables:
%%% mode: latex
%%% TeX-master: t
%%% End:
