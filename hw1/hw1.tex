\documentclass{abrice}

\title{Comp 462: Homework 1}
\author{Anthony Brice}

\tikzstyle[scale] = [regular polygon, regular polygon sides=3, draw=black, ]

\tikzset{
  scaler/.style={draw=black, isosceles triangle, minimum
    height=4.5em},
  node rotated/.style = {rotate=270},
  border rotated/.style = {shape border rotate=270},
  cosiner/.style = {draw=black, rectangle, minimum height=4.5em, minimum
    width=5em}
}

\begin{document}
\maketitle

\section{Exercise 2.1}
\begin{enumerate}[label= (\alph*)]
\item Since we have a homogeneous linear differential equation, any linear combination of solutions
  is also a solution. Consider $\ddot{y}(t) = - {\omega_0}^2
  y(t)$. Then
  \begin{alignat*}{2}
    && 0 &= D^2(y) + {\omega_0}^2 D^0(y) \\
    && &= (D^2 + {\omega_0}^2 D^0)(y) \\
    && &= (D^2 + {\omega_0}^2)(y) \\
    && &= (D + i \omega_0)(D - i \omega_0)(y) \\
    &\Rightarrow \quad && y = \ e^{-i \omega_0 t}\quad \textrm{and}\quad y = e^{i
      \omega_0 t} \\
    &\Rightarrow \quad && y(t) = \sin(\omega_0 t)\, .
  \end{alignat*}
\item The initial displacement is $y(0) = \cos(\omega_0 \cdot 0) = 1$.
\item \hfill
  \begin{figure}
    \caption{An actor model of a tuning fork. The initial displacement is given
      by the parameter $i$.}
    \centering
    \begin{tikzpicture}[node distance=12em,>=triangle 90, thick, ->]
      \node (a) [scaler, label={[label distance=0.65em]\sffamily \small Scale}] {$\omega_0$};
      \node (b) [cosiner, right of=a, label={\sffamily \small Cosiner}] {$\cos{}
        \quad i$};
      \node (a-left) [coordinate,left of=a] {};

      \node[fit=(a) (b), draw, inner ysep=2.3em, inner xsep=3.5em,
      label={145:\sffamily \small Tuning Fork}] {};

      \draw +(-2,0) -- node [anchor=south east, pos=0.9] {$x(t)$} (a);
      \draw [>->] (a) -- (b);
      \draw [>-] (b) -- node [anchor=south west, pos=0.1] {$y(t)$} +(2.3,0);
    \end{tikzpicture}
  \end{figure}
\end{enumerate}

\section{Exercise 2.3}

\begin{enumerate}[label= (\alph*)]
\item \textit{Claim.}
  $\dot{\theta}_y(t) = \dot{\theta}_y(0) + {1 \over I_{yy}} \int_0^t T_y(\tau)\,
  d\tau$ is linear if and only if $\dot{\theta}_y(0) = 0$.

  STUB
  % \begin{proof} Assume that $\dot{\theta}_y(0) = 0$. Then
  %   \begin{align*}
  %     \dot{\theta}_{ay_1 + by_2}(t)
  %     &= {1 \over I_{yy}} \int_0^t T_{ay_1}(\tau) + T_{by_2}(\tau)\, d\tau \\
  %     &= {1 \over I_{yy}} \left( \int_0^t T_{ay_1} (\tau)\, d\tau + \int_0^t
  %       T_{by_2} (\tau)\, d\tau \right) \\
  %     &= \dot{\theta}_{ay_1}(t) + \dot{\theta}_{by_2}(t)\, .
  %   \end{align*}

  %   Now
  % \end{proof}
\item \textit{Claim.} The cascade of any two linear actors is linear.
  \begin{proof}
    Consider that actors are only a special class of functions, and that the
    composition of linear functions is linear. Since to cascade actors is to
    compose their functions, the cascade of any two linear actors is linear.
  \end{proof}
\item Consider system $S\colon X^2 \rightarrow Y$. Augmenting for linearity
  yields
  \begin{align*}
    S(ax_1 + bx_2, cx_3 + dx_4)
    &= S(ax_1,cx_3 + dx_4) + S(bx_2,cx_3 + dx_4) \\
    &= S(ax_1,cx_3) + S(ax_1,dx_4) + S(bx_2,cx_3) + S(bx_2,dx_4) \\
    &= acS(x_1, x_3) + adS(x_1, x_4) + bcS(x_2, x_3) + bdS(x_2, x_4)\, .
  \end{align*}

  STUB
\end{enumerate}

\section{Exercise 2.5}

STUB

\section{Exercise 7.1}

\textit{Claim.} The composition $f \circ g$ of two affine function $f$ and $g$
is affine.

\begin{proof}
  Consider two affine functions: $f(x(t)) = ax(t) + b$ and
  $g(x(t)) = cx(t) + d$. Then
  \begin{align*}
    f \circ g (x(t))
    &= a(cx(t) + d) + b \\
    &= acx(t) + d + b\, .
  \end{align*}
  Then we have a proportionality constant of $ac$ and a bias of $d + b$, so the
  composition is affine.
\end{proof}

\section{Exercise 7.2}

\begin{enumerate}[label= (\alph*)]
\item
\end{enumerate}

\end{document}

%%% Local Variables:
%%% mode: latex
%%% TeX-master: t
%%% End:
