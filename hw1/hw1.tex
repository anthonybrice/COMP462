\documentclass{abrice}

\title{Comp 462: Homework 1}
\author{Anthony Brice}

\begin{document}
\maketitle

\section{Exercise 2.1}

\begin{enumerate}[label=\alph*)]
\item embedded -- the quality of being hidden.
\item device driver -- a set of software functions that facilitate the use of
  an I/O port.
\item top down -- a methodology describing design as a cyclic process, beginning
  with a problem statement and ending up with a solution.
\item call graph -- a graphical way to define how the software/hardware modules
  interact.
\item data flow graph -- a block diagram of a system, showing the flow of
  information.
\item successive refinement -- a technique to convert a problem statement into a
  software algorithm consisting of iteratively decomposing the task into a set
  of simpler subtasks.
\item dynamic efficiency -- a measure of how fast a program executes via seconds
  or processor bus cycles.
\item static efficiency -- the number of memory bytes a program requires in
  execution.
\item real time -- a property of certain computer systems in which an upper
  bound is placed on the time required to perform the input-calculation-output
  sequence.
\end{enumerate}

\section{Exercise 2.3}

A requirement describes a specific parameter that a proposed system must
satisfy, like the ability to mount on a car dash. A specification is a detailed
parameter describing how a system should work, such as giving the exact size and
weight of a device. A constraint is a limitation within which a system must
operate, such as a restriction on available materials.

\section{Exercise 2.5}

Three operations software on a microcomputer embedded in a vending machine must
perform include: summing input currency, calculating and dispensing change, and
dispensing a user's specific selection among a variety.

\section{Exercise 7.1}

To say a function is public is to indicate that other modules may invoke it,
whereas to say a function is private indicates that only the module in which it
is declared should use it. This distinction is important because private methods
often lack documentation, serve no purpose towards the module's intended use,
and can even alter a module's state in ways that introduce bugs. For example, a
user of a dictionary might find the functions in Listing~\ref{lst:dict-public}
useful, so they would be made public. Implementation-specific functions---such
as a possible hash function, declared in Listing~\ref{lst:dict-private}, for a
hash table-based implementation---have no reason to made public.

\begin{listing}
\caption{The public function declarations of a dictionary module.}
\label{lst:dict-public}
\begin{minted}{c}
Dictionary* init(); // Constructor
int insert(Dictionary* self, A* key, B* value);
B* lookup(Dictionary* self, A* key);
\end{minted}
\end{listing}

\begin{listing}
\caption{A private hash function for a dictionary.}
\label{lst:dict-private}
\begin{minted}{c}
int hash(void* key);
\end{minted}
\end{listing}

\section{Exercise 7.2}

To say a variable is public is to indicate that other modules may read and write
to the variable, whereas to say a variable is private indicates that only the
module in which it is declared should use it. This distinction is important
because private variables very often contain stateful information sensitive to
outside mutation. For example, an instance of an array implementation would
likely have a private variable storing the number of elements in the array at
any moment in time. Increasing the variable from another module could cause the
program to access unallocated, insecure memory, and likewise decreasing the
variable could cause the program to leak memory.

\end{document}

%%% Local Variables:
%%% mode: latex
%%% TeX-master: t
%%% End:
