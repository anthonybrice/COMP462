\documentclass{abrice}

\title{Comp 462: Homework 1}
\author{Anthony Brice}

\begin{document}
\maketitle

\section{Exercise 2.1}
\begin{enumerate}[label=(\alph*)]
\item Yes, other solutions exist. Consider $\ddot{y}(t) = - {\omega_0}^2
  y(t)$. Then
  \begin{alignat*}{2}
    && D^2(y) + {\omega_0}^2 D^0(y) &= 0 \\
    && (D^2 + {\omega_0}^2 D^0)(y) &= 0 \\
    && (D^2 + {\omega_0}^2)(y) &= 0 \\
    && (D + i \omega_0)(D - i \omega_0)(y) &= 0 \\
    &\Rightarrow \quad & y = e^{-i \omega_0 t}\quad \textrm{and}\quad &y = e^{i
      \omega_0 t} \\
    &\Rightarrow \quad & y(t) = \sin(\omega_0 t)\, .
  \end{alignat*}
\item The initial displacement is $y(0) = \cos(\omega_0 \cdot 0) = 1$.
\item STUB
\end{enumerate}

\section{Exercise 2.3}

\begin{enumerate}[label= (\alph*)]
\item \textit{Claim.}
  $\dot{\theta}_y(t) = \dot{\theta}_y(0) + {1 \over I_{yy}} \int_0^t T_y(\tau)\,
  d\tau$ is linear if and only if $\dot{\theta}_y(0) = 0$.

  STUB
  % \begin{proof} Assume that $\dot{\theta}_y(0) = 0$. Then
  %   \begin{align*}
  %     \dot{\theta}_{ay_1 + by_2}(t)
  %     &= {1 \over I_{yy}} \int_0^t T_{ay_1}(\tau) + T_{by_2}(\tau)\, d\tau \\
  %     &= {1 \over I_{yy}} \left( \int_0^t T_{ay_1} (\tau)\, d\tau + \int_0^t
  %       T_{by_2} (\tau)\, d\tau \right) \\
  %     &= \dot{\theta}_{ay_1}(t) + \dot{\theta}_{by_2}(t)\, .
  %   \end{align*}

  %   Now
  % \end{proof}
\item \textit{Claim.} The cascade of any two linear actors is linear.
  \begin{proof}
    Consider that actors are only a special class of functions, and that the
    composition of linear functions is linear. Since to cascade actors is to
    compose their functions, the cascade of any two linear actors is linear.
  \end{proof}
\item Consider system $S\colon X^2 \rightarrow Y$. Augmenting for linearity
  yields
  \begin{align*}
    S(ax_1 + bx_2, cx_3 + dx_4)
    &= S(ax_1,cx_3 + dx_4) + S(bx_2,cx_3 + dx_4) \\
    &= S(ax_1,cx_3) + S(ax_1,dx_4) + S(bx_2,cx_3) + S(bx_2,dx_4) \\
    &= acS(x_1, x_3) + adS(x_1, x_4) + bcS(x_2, x_3) + bdS(x_2, x_4)\, .
  \end{align*}
\end{enumerate}

\section{Exercise 2.5}

STUB

\section{Exercise 7.1}

\textit{Claim.} The composition $f \circ g$ of two affine function $f$ and $g$
is affine.

\begin{proof}
  Consider two affine functions: $f(x(t)) = ax(t) + b$ and
  $g(x(t)) = cx(t) + d$. Then
  \begin{align*}
    f \circ g (x(t))
    &= a(cx(t) + d) + b \\
    &= acx(t) + d + b\, .
  \end{align*}
  Then we have a proportionality constant in $ac$ and a bias in $d + b$, so the
  composition is affine.
\end{proof}

\section{Exercise 7.2}

\begin{enumerate}[label= (\alph*)]
\item
\end{enumerate}

\end{document}

%%% Local Variables:
%%% mode: latex
%%% TeX-master: t
%%% End:
