\documentclass{abrice}

\title{Comp 462: Homework 1}
\author{Anthony Brice}

\usepackage{siunitx}
\sisetup{
  %math-rm=\mathsf,
  %text-rm=\sffamily
}

\tikzstyle[scale] = [regular polygon, regular polygon sides=3, draw=black, ]

\tikzset{
  scaler/.style={draw=black, isosceles triangle, minimum
    height=4.5em},
  node rotated/.style = {rotate=270},
  border rotated/.style = {shape border rotate=270},
  cosiner/.style = {draw=black, rectangle, minimum height=4.5em, minimum
    width=5em}
}

\begin{document}
\maketitle

\section{Exercise 2.1}
\begin{enumerate}[label= (\alph*)]
\item Since we have a homogeneous linear differential equation, any linear combination of solutions
  is also a solution. Consider $\ddot{y}(t) = - {\omega_0}^2
  y(t)$. Then
  \begin{alignat*}{2}
    && 0 &= D^2(y) + {\omega_0}^2 D^0(y) \\
    && &= (D^2 + {\omega_0}^2 D^0)(y) \\
    && &= (D^2 + {\omega_0}^2)(y) \\
    && &= (D + i \omega_0)(D - i \omega_0)(y) \\
    &\Rightarrow \quad && y = \ e^{-i \omega_0 t}\quad \textrm{and}\quad y = e^{i
      \omega_0 t} \\
    &\Rightarrow \quad && y(t) = \sin(\omega_0 t)\, .
  \end{alignat*}
\item The initial displacement is $y(0) = \cos(\omega_0 \cdot 0) = 1$.
\item \hfill
  \begin{figure}
    \caption{An actor model of a tuning fork. The initial displacement is given
      by the parameter $i$.}
    \centering
    \begin{tikzpicture}[node distance=12em,>=triangle 90, thick, ->]
      \node (a) [scaler, label={[label distance=0.65em]\sffamily \small Scale}]
      {$\omega_0$};
      \node (b) [cosiner, right of=a, label={\sffamily \small Cosiner}] {$\cos{}
        \quad i$};
      \node (a-left) [coordinate,left of=a] {};

      \node[fit=(a) (b), draw, inner ysep=2.3em, inner xsep=3.5em,
      label={145:\sffamily \small Tuning Fork}] {};

      \draw +(-2,0) -- node [anchor=south east, pos=0.9] {$x(t)$} (a);
      \draw [>->] (a) -- (b);
      \draw [>-] (b) -- node [anchor=south west, pos=0.1] {$y(t)$} +(2.3,0);
    \end{tikzpicture}
  \end{figure}
\end{enumerate}

\section{Exercise 2.3}

\begin{enumerate}[label= (\alph*)]
\item \textit{Claim.}
  System $S(T_y) = \dot{\theta}_y(t) = \dot{\theta}_y(0) + {1 \over I_{yy}} \int_0^t
  T_y(\tau)\, d\tau$ is linear if and only if $\dot{\theta}_y(0) = 0$.

  \begin{proof}
    Assume $\dot{\theta}_y(0) = 0$. Then
    \begin{align*}
      S(aT_{y1} + bT_{y2}) = \dot{\theta}_y(t)
      &= {1 \over I_{yy}} \int_0^t a T_{y1}(\tau) + b T_{y2}(\tau)\, d\tau \\
      &= {a \over I_{yy}} \int_0^t T_{y1}(\tau)\, d\tau + {b \over I_{yy}}
        \int_0^t T_{y2}(\tau)\, d\tau \\
      &= a S(T_{y1}) + b S(T_{y2})\, .
    \end{align*}
    Then if $\dot{\theta}_y(0) = 0$, $S(T_y)$ is linear.

    Now assume $S(T_y)$ is linear and for purposes of contradiction that
    $\dot{\theta}_y(0) = C \neq 0$. Then
    \begin{align*}
      S(aT_{y1} + bT_{y2}) = \dot{\theta}_y(t)
      &= C + {1 \over I_{yy}} \int_0^t a T_{y1}(\tau) + b T_{y2}(\tau)\, d\tau
      \\
      &= C + {a \over I_{yy}} \int_0^t T_{y1}(\tau)\, d\tau + {b \over I_{yy}}
        \int_0^t T_{y2}(\tau)\, d\tau \\
      &\neq 2C + {a \over I_{yy}} \int_0^t T_{y1}(\tau)\, d\tau + {b \over I_{yy}}
        \int_0^t T_{y2}(\tau)\, d\tau \\
      &= a S(T_{y1}) + b S(T_{y2})\, .
    \end{align*}
    Clearly we have a contradiction. Then $S(T_y)$ is linear only if
    $\dot{\theta}_y(0) = 0$.
  \end{proof}
\item \textit{Claim.} The cascade of any two linear actors is linear.
  \begin{proof}
    Consider that actors are only a special class of functions, and that the
    composition of linear functions is linear. Since to cascade actors is to
    compose their functions, the cascade of any two linear actors is linear.
  \end{proof}
\item Augmenting the definition of superposition such that it applies to actors
  of two input signals and one output signal yields
  $S(ax_{i1} + bx_{i2}, ax_{j1} + bx_{j2}) = aS(x_{i1},x_{j1}) +
  bS(x_{i2},x_{j2})$.

  \textit{Claim.} System $S(x_i,x_j) = y(t) = x_i(t) + x_j(t)$ is linear.

  \begin{proof}
    \begin{align*}
      S(ax_{i1} + bx_{i2}, ax_{j1} + bx_{j2})
      &= ax_{i1} + bx_{i2} + ax_{j1} + bx_{j2} \\
      &= a(x_{i1} + x_{j1}) + b(x_{i2} + x_{j2}) \\
      &= aS(x_{i1},x_{j1}) + bS(x_{i2},x_{j2})\, .
    \end{align*}
  \end{proof}
\end{enumerate}

\section{Exercise 2.5}

\begin{enumerate}[label= (\alph*)]
\item $\theta(t) = \theta(0) + \int_0^t \dot{\theta}(\tau)\, d\tau$.
\item No, the system is unstable. Let input $\dot{\theta}$ be the unit step;
  then the input is bounded, but the output is unbounded.
\item
  \begin{align*}
    \theta(t)
    &= \int_0^t \dot\theta (\tau)\, d\tau \\
    &= K \int_0^t \psi(\tau) - \theta(\tau)\, d\tau \\
    &= K \int_0^t a\, d\tau - K \int_0^t \theta(\tau)\, d\tau \\
    &= Kat - K \int_0^t \theta(\tau)\, d\tau \\
    &= au(t)(1 - e^{-Kt})\, .
  \end{align*}
  At $t = 0$, $\theta(0) = 0$. As $t$ gets large, $\theta$ approaches $a$.
\end{enumerate}

\section{Exercise 7.1}

\textit{Claim.} The composition $f \circ g$ of two affine function $f$ and $g$
is affine.

\begin{proof}
  Consider two affine functions: $f(x(t)) = ax(t) + b$ and
  $g(x(t)) = cx(t) + d$. Then
  \begin{align*}
    f \circ g (x(t))
    &= a(cx(t) + d) + b \\
    &= acx(t) + d + b\, .
  \end{align*}
  Then we have a proportionality constant of $ac$ and a bias of $d + b$, so the
  composition is affine.
\end{proof}

\section{Exercise 7.2}

\begin{enumerate}[label= (\alph*)]
\item The sound pressure of the loudest sound is given by $H$ in
  \begin{equation*}
    \SI{100}{\decibel} = 20 \log_{10}\left( {H \over \SI{20}{\micro\pascal}}
    \right).
  \end{equation*}
  Thus $H = \SI{2}{\pascal}$.
\item If the maximum pressure is \SI{2}{\pascal} and the smallest difference is
  \SI{20}{\micro\pascal}, then $\SI{2}{\pascal} / \SI{20}{\micro\pascal} =
  \SI{100000}{}$ is the largest number we need to encode in binary. Thus we need
  at least $17$ bits.
\end{enumerate}

\end{document}

%%% Local Variables:
%%% mode: latex
%%% TeX-master: t
%%% End:
